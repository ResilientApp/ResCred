\section{Milestone Review}
In this section, we discuss the planned milestones for the past few weeks, whether or not they are completed/in progress/not started, our progress with the milestone so far (if not completed), and any setbacks we encountered.

\subsection{Assigned to all}

\subsubsection{Architectual Design}
\boxed{\texttt{STATUS: COMPLETED}}

It is critical to agree on a common architecture before starting the code. A well-designed architecture promotes development time and avoids potential pitfalls, redesigns, and security compromises. In the context of ResCred, a good architecture can prevent vulnerabilities in the credential issuance and verification process, as well as ensuring transparent access to credentials and ease of use for the general public.

Accounting for our technical and resource limitations, our team agreed on an architecture where a React frontend interacts with the smart contract GraphQL backend, which in turn interacts with the ResilientDB smart contract service. There will be a single instance of a Registry contract that holds all issuing bodies and credentials, with which the frontend can call various contract functions to perform ResCred functionalities such as creating issuing body, creating a new smart contract-based credential, granting the credential to an address, request verification, and so on. Thanks to the features of Solidity, we do not need a separate backend for the purposes of ResCred.

\subsubsection{Division of Labor}
\boxed{\texttt{STATUS: COMPLETED}}

To ensure fair and even distribution of labor, we aim to assign various tasks to all five members of the team. All tasks shall be assigned such that each person’s task list takes a similar amount of time, effort, and resources for each person. The tasks should also be reasonably independent, such that each person can work on their assigned tasks without being affected by the progress of one another.

Although it is difficult to ensure that everyone gets the exact same amount of work, we have attempted to divide labor by components (e.g., frontend for Issuing Body, Credential Holder, and Verifier, GraphQL modification, and smart contract development). As such, task dependency has been minimized and team members may work on their tasks asynchronously and collaborate using GitHub and Discord as necessary.

\subsubsection{GraphQL Schema and Smart Contract Function Suggestion}
\boxed{\texttt{STATUS: IN PROGRESS}} 

As team members responsible for frontend implement their share of UI, they may realize various
endpoints needed to fetch information to display or update information about parties involved. For
example, the team member responsible for Verifier UI may realize that there is a need for a
\verb|getApprovedRequests()| function that returns a list of past approved requests to enable easy tracking of past requests and facilitate speedy access to the credential holder over the blockchain.

The smart contract development process is constantly informed by frontend team members as we work on integrating the frontend with GraphQL API and the smart contract in development. This “milestone” will serve as a constant reminder to inform and collaborate with each other as new demands for the Registry and Credential smart contract functions arise.

\subsection{Federico: Credential Holder UI}

\subsubsection{Owned Credential List}
\boxed{\texttt{STATUS: COMPLETED}}

This milestone aimed to Implement a dashboard where Credential Holders can access a list of their credentials . Metadata like the credential name, holder name, issuer name, and domain are included with every credential. This feature improves openness and usability by guaranteeing that Credential Holders get a comprehensive overview of their credentials. This list is an essential tool for Credential Holders to manage their credentials in the ResCred environment, setting the stage for future communications with issuing organizations and verifiers.

This was accomplished by creating the CredentialHolderDashboard.vue component, which displays individual credentials and is supported by CredentialCard.vue and confirmModal.vue. I made sure the layout was responsive and aesthetically pleasing by using BootstrapVue3 components from the internet. By adding fictitious data and using computed attributes to divide credentials into "Pending" and "Approved" portions, the list became dynamic.

At first, I encountered routing problems that made it impossible for the dashboard to appear properly under the /dashboard route. Examining the router settings in detail and making sure the CredentialHolderDashboard.vue file was imported appropriately were necessary steps in debugging these problems. Additionally, it was necessary to use the page-container class and experiment with CSS in order to style the dashboard to match the project theme's background. Making the credential list dynamic rather than hardcoded presented another difficulty. In order to manage state updates, Vue.js calculated properties required a considerable learning curve. The milestone was accomplished effectively in spite of these obstacles, and it established a strong basis for future functionality.

\subsubsection{Track Pending Verification Requests}
\boxed{\texttt{STATUS: COMPLETED}}

This milestone necessitated making sure that actions are verified before they take effect and presenting "Approve" and "Reject" buttons for pending credentials. This feature is essential for Credential Holders to maintain control over their credentials, avoid unwanted verifications, and communicate with verifiers in a smooth manner.

I did this by integrating ConfirmationModal.vue into the dashboard workflow and developing it to manage user confirmations. The handleConfirm function modified the credential's state based on the user's confirmation, while the openConfirmModal function dynamically configured the modal's title and message. Making sure the modal appeared at the appropriate moment was a significant challenge as I'm new to frontend. At first, commands such as "Approve" and "Reject" altered credential statuses directly, avoiding the popup. Restructuring the workflow to make sure the modal opened before any status updates took place was necessary to debug this. To make sure the modal displays the right information for each operation, another problem was correctly transmitting data between components (CredentialHolderDashboard.vue, CredentialCard.vue, and ConfirmationModal.vue). These challenges forced me to dive deeper into Vue.js event handling and parent-child communication, which improved my understanding of the framework. Despite these difficulties, the feature was successfully completed, enabling Credential Holders to manage their credentials efficiently.

\subsubsection{Credential Import}
\boxed{\texttt{STATUS: IN PROGRESS}}

This milestone aims to put in place a functionality that enables Credential Holders to safely import credentials into the system. In order to maintain confidentiality and privacy, imported credentials will be stored in a passphrase-protected manner like ResWallet. A credential's validity will be checked by the system using GraphQL to make sure the credential is present on the connected smart contract. Credential Holders can now incorporate external credentials into the ResCred system for unified administration thanks to this functionality, which increases user freedom to my knowledge.

Due to the significant amount of work needed to finish the basic user interface for Credential Holders, this milestone has not yet been completed. Furthermore, backend integration and our progress on developing GraphQL endpoints are prerequisites for the import capability. Working closely with the group to ensure the security of imported credentials fits with the smart contract architecture will also be necessary. This milestone's commencement was delayed by these requirements and the challenging learning curve associated with using Vue.js. But now that the basic user interface is in place, I'm better equipped to tackle this task in the upcoming weeks.

\subsection{Vikram: Issuing Body UI}
\subsubsection{Register a New Smart Contract Credential}
\boxed{\texttt{STATUS: COMPLETED}}

In order for the issuing body to create credentials, there must be a frontend UI which allows them to create smart contracts with configurable parameters to specify certain information on how the credentials are used (i.e., lifespan). These credentials are then signed and hashed, stored on hash blockchain, and a link is sent to the credential holder. This process is important, because it gives the credential holder access to the service they need with highly customizable parameters through smart contracts.

Currently, just the UI has been done for this portion, with a create smart contract section for specifying parameters and name of smart contracts. This portion still needs to be integrated with the ResilientDB infrastructure and contract service to send data to be stored on the hash blockchain. 

\subsubsection{Manage Credentials}
\boxed{\texttt{STATUS: IN PROGRESS}}

The issuing body must be able to see the list of smart contracts they own and associated credentials, as well as relevant metadata for each, and the ability to revoke contracts given a reason. They must also be able to see a list of pending verification requests and provide users with the ability to approve or deny these via cryptographic signature. These features are important because the issuing body needs control over who accesses their service, and giving a list of associated credentials is a way to keep credential holders in check, with the option to revoke them at any time. On the user side, the ability to approve or deny requests via signature is important since users need a secure way to access services via ResCred.

Currently, just the UI has been done for this portion, with a section for users to approve or deny incoming requests. This portion still needs to be integrated with the ResilientDB infrastructure and contract service to fetch data from the hash blockchain on credentials created by the issuing body, and credentials owned by credential holders.

\subsubsection{GraphQL Service Integration}
\boxed{\texttt{STATUS: IN PROGRESS}}

Integrate smart contract deployment with GraphQL service, utilizing the developed smart contract from Zhenkai. All Issuing Body rely on the same contract, only passing in different values to the contract constructor depending on the Issuing Body’s needs. An Issuing Body can technically intercept and modify the contract being deployed, but at their own risk. Addresses of smart contracts must be stored in the browser's local storage. This is important, because the actual functionality of developing smart contracts and storing credentials on the hash blockchain must be done for this website to work.


\subsection{Ethan: Smart Contract GraphQL Service}
\subsubsection{Address Ownership Verification}
\boxed{\texttt{STATUS: IN PROGRESS}}

I have confirmed that address ownership does not exist in the contract service at the current moment. I have been combing through the source code to establish the current status of the contract service and GraphQL service. I am currently designing a schema to allow for generating the address ownership fields as well as allow for users to put in their private keys from the Web UI to sign the smart contracts. I believe implementing the scheme will not take long once I have properly identified all the necessary components to the GraphQL and ResilientDB structure. 

\subsubsection{ResContract CLI Testing}
\boxed{\texttt{STATUS: IN PROGRESS}}

At the current moment, I am able to confirm that the smart contract is \textit{in theory} workable and able to create smart contracts. Zhenkai and I are continuing to work on modifying the underlying source code to ensure all fields that are needed within the ResCred are fully implemented. We have yet to identify any significant bugs within the ResContract CLI, though we notice some interesting code styles in the code base. Professor Sadoghi had mentioned that there was a new update to the ResContract CLI tooling and we have also been exploring the new changes and determining whether that affects our current work. As of now, we believe that those changes help link the tool to the ResWallet application and do not impact our current goals. However, we will continue to investigate, submit, and deploy patches to the CLI service as needed in the future. We believe there are certain missing elements that would be beneficial to the tool (see above) and may help out future project developers.

\subsection{Zhenkai: Smart Contract Development}
\subsubsection{Solidity Research}
\boxed{\texttt{STATUS: IN PROGRESS}}

As a cornerstone of the project’s success, we designate Solidity smart contracts as the backend. This presents some advantages and risks. With a syntax reminiscent of C, we envision that Solidity enables us to take advantage of the existing ResilientDB fabric as a distributed backend for our ResCred project, such that we do not have to spend precious development time on setting up a backend (or additional off-chain database). That said, we are not aware of the precise capabilities of Solidity apart from glimpses at the syntax and limited practical experience with the ResilientDB blog tutorials. This gives us reason to investigate the capabilities and limitations of Solidity smart contracts.

After some investigation, we can conclude that Solidity is a reasonably featureful programming
language specifically designed for the Ethereum Virtual Machine (EVM). Apart from including common
features such as functions and control flow, Solidity provides additional features built on top of
contracts and blockchain transactions. This is beneficial for ResCred as all actions taken by each
party (Issuing Body, Credential Holder, and Verifier) is transparently recorded on the blockchain,
such that if an investigation arises, the contract deployed on ResilientDB can serve as an immutable
forensic evidence. As we continued our development, we noticed that Solidity has many limitations
especially when it comes to more complex storage—there are no resizable arrays, \verb|structs| with mappings cannot be passed around directly, etc. However, we believe that these limitations can be worked around given enough time.

\subsubsection{Team-wide Shared Infrastructure Setup}
\boxed{\texttt{STATUS: COMPLETED}}

To improve development experience (DX) and accommodate for uneven distribution of infrastructure skills, we have decided to set up a shared, self-hosted instance of ResilientDB KV and Contract Service. Instead of having each member go through the same process of setting up local instances of ResilientDB, we would only have to do this once, saving precious development time and ensuring a single source of truth for project deployment and testing.

We have prepared a Proxmox VE machine, a Debian-based hypervisor platform based on KVM, to run ResilientDB services 24/7. All team members are granted SSH access to the virtual machine hosting the ResilientDB services. In addition, thanks to the Netbird VPN, team members do not have to be on the same LAN to access the virtual machine, reducing development friction and improving testing experience. Team members also have the ability to modify source code and add debugging statements to the services as necessary.

\subsubsection{Project Repository Setup}
\boxed{\texttt{STATUS: COMPLETED}}

The initial repository setup can be a complicated process. For a TypeScript-based project, this typically involves choosing a package manager (npm, pnpm, bun runtime, yarn, etc), setting up TypeScript, selecting frontend UI framework, component libraries, linting, formatting, etc. By ensuring that the initial repository contains all of these components, team members have a consistent library version and coding style, and can prevent code friction caused by code reviews of minor issues such as version mismatch and inconsistent indentation.

Thankfully, the ResilientDB team graciously provided a create-resilient-app command available through npx for developers to easily scaffold a ResilientDB-based frontend, solving the TypeScript configuration and framework selection challenges. Using this, we set up a shared repository with additional niceties such as .gitignore and project-wide type-informed linting using ESLint.

\subsubsection{Smart Contract Design and Development}
\boxed{\texttt{STATUS: IN PROGRESS}}

ResCred’s architecture places Solidity smart contracts as a backend built on top of ResilientDB contract service. Our goal is to have it support all of the frontend’s demands without involving an external NodeJS backend or database. Therefore, the smart contract must be able to efficiently store all of issuing bodies, credentials, and be able to track pending and past verification requests, exposed via smart contract functions. Security considerations must be in place as well—only the credential holder may see and approve verification requests, expired credentials may not be used to approve requests, only the issuing body can grant a credential, etc.

We have settled on two main contracts called \verb|Registry| and \verb|Credential|, along with a
struct called \verb|IssuingBody|. \verb|Registry| maintains a mapping of owner address to
\verb|IssuingBody| and provides a \verb|register| function so that Issuing Body UI may create an
issuing body with an issuing body name and domain (the domain is used by the frontend to verify the
contract owner address’s authenticity; the domain must keep the contract address at
https://<domain>/.well-known/<issuing-body-name>; this is subject to change). The \verb|Registry|
also provides a function to create a new \verb|Credential| given that \verb|msg.sender| has already registered as an issuing body.

On the other hand, \verb|Credential| represents a single credential (whether a diploma or vocational license). An instance of the contract tracks all credential holders and current or past verification requests. \verb|Credential| exposes functions to create verification request (as a verifier), get verification request (as a credential holder), approve verification request (as a credential holder), and query status of verification request (as a verifier). Security and sanitation checks are in place to prevent invalid requests (verifier approving a verification request, for example, is forbidden because a user cannot approve a request for anyone other than themselves).

While the overarching ideas behind the smart contracts have been implemented as Solidity code, some
language limitations remain in place that prevent successful usage of the contract. One of the most
important limitations we overlooked is that Solidity has no resizable array (e.g., C++’s
\verb|std::vector|) by default. We have found alternatives (such as using a mapping or third-party
libraries for linked lists), but that requires some additional time for adjusting the codebase. In
addition, currently we emit events for all actions successfully performed by all parties, though
ResilientDB contract does not expose methods for tracking these events. We have made adjustments
such that the code does not rely on these events, but this makes the contracts store a significant
amount of data (which, since they are not stored on the actual Ethereum chain, has no per-write gas
cost fortunately). Solidity also presents issues when it comes to mappings storage, preventing us
from calling functions on \verb|Credential| directly, which may mean that we have to expose
\verb|Credential| API through \verb|Registry| instead. We plan to address these issues in the upcoming weeks and ensure the timely release of ResCred.

\subsection{Avni: Verifier UI}
\subsubsection{Submit Verification Request}
\boxed{\texttt{STATUS: \textit{unknown}}}

\subsubsection{Track Status of Verification Request}
\boxed{\texttt{STATUS: \textit{unknown}}}

\subsubsection{Integrate Verifier UI with Smart Contract GraphQL API}
\boxed{\texttt{STATUS: \textit{unknown}}}

\section{Other Challenges and Issues}
One significant issue for us is coordination and communication about the architecture and design of the project to ensure all team members stay in sync and integration of different components is doable without a significant amount of work. If we are all on the same page with the architecture, integration across various full-stack components will be easier since front-end members also have to implement logic that properly fits into the backend architecture. More consistent communication about milestones and progress needs to happen for this to be achieved, as well as in-person discussions about project design.

\section{Next Steps}
Apart from accomplishing our current milestones, our next steps involve integrating the smart contract GraphQL API with the frontend, refining our own Credential and Registry smart contracts, implementing domain ownership verification for an issuing body’s owner address, investigating additional features, and conducting more rigorous end-to-end testing.


