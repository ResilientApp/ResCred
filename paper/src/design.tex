\section{Design}

We propose the following design for CredRes:
\begin{enumerate}
    \item The Credential Holder, either a natural or legal person, undergoes the certification process as stipulated by the Issuing Body, creates a DID (Decentralized Identifier) if they don’t have one already, and lastly reports to the Issuing Body with their DID.
    \item The Issuing Body (e.g., government agencies, training programs, and educational institutions) maintains replicas of the blockchain, and creates credentials in the form of smart contracts. An Issuing Body can “mint” new credentials for the Credential Holder, vouching for the attainment with its own DID. The nature of credentials as smart contracts also allows for expiry of such contracts to facilitate transparent recertification processes, but only if it is specified in the contract code at the time of issuing.
    \item Lastly, the Credential Holder must accept the credential using their own DID. Once the DID is associated with the credential, employers and other third parties can verify DID ownership using challenge-response mechanisms that request the DID owner to sign a message of the verifier’s choosing.
    \item To reiterate, smart contracts are used throughout the process to manage the lifecycle of certifications—from the initial issuance, to expiry and revocation (with metadata within each transaction for convenient auditing).
\end{enumerate}

