\section{Definitions}
We would like to clarify some terms we used in our project that have caused some misunderstanding
from our proposal.

\begin{enumerate}
    \item Credentials \\
        In the scope of the project, credentials describe the qualifications an individual may hold, including academic degrees, certifications, licenses, etc.
        This does not apply to identification credentials like passports, driver\'s license, birth
        certificates, etc.
    \item Issuing Body \\
        The organization that issues credentials to Credential Holders. The authenticity of the public key associated with the Issuing Body can be verified via the 
        {did-configuration}\footnote{\url{https://identity.foundation/.well-known/resources/did-configuration/}} well-known resource served from the Issuing Body’s domain name. There are no constraints on requirements to become an Issuing Body. Verifiers are responsible for determining the reputation of the organization and domain name.
    \item Transmission \\
        The act of transferring the credential. The method in which the Issuing Body transmits the issued credential to the recipient is not strictly stipulated, though it is recommended that they only transmit the credential through a secured and authenticated channel, such as a web portal (e.g., universities can issue diplomas on a authenticated issuance portal tied to the student’s email address).
    \item Verification \\
        When presented with a credential (e.g., degree diploma or license), the Verifier (such as an employer) may create a verification request via the associated smart contract that can only be approved (cryptographically signed) or rejected by the credential owner. This way, public credential information cannot be used except by the actual owner, preventing identity theft.
\end{enumerate}

