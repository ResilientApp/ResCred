\section{Technology Stack}
\begin{enumerate}
    \item Web Application: Typescript, CSS, Vue, Vite 
    \item Backend: ResilientDB GraphQL server, Solidity
    \item Database: Powered by ResilientDB Smart Contract Service
\end{enumerate}

\section{Architecture}

\subsection{Frontend}

ResCred frontend is a Vite-compiled static site that has the ability to do HRM/live reload during development thanks to Vite (npm run dev). Due to the Same Origin Policy, we do not want to serve the GraphQL server on a different origin, so we serve both the static frontend and the GraphQL server on a single Nginx reverse proxy. 
We set \verb|/| to redirect to the frontend, and \verb|/graphql| to redirect to the GraphQL API,
which talks to ResContract CLI and the smart contract service behind the firewall.

\subsection{Backend}

Thanks to the power of the Ethereum Virtual Machine, ResCred’s real backend is technically made up of smart contracts, accessed via several layers of indirection: GraphQL API, ResContract CLI, and smart contract service.

Before the app goes online, a version of the \verb|Registry| contract needs to be deployed using
\verb|rescontract deploy|. The resulting contract address is embedded in the frontend
\verb|RegistryClient| class and subclasses. The \verb|Registry| helps create and keep track of
various entities involved in credential management as separate smart contracts. For instance, the
two organization types each have their own smart contracts, \verb|IssuingBody| and \verb|Verifier|,
to ensure open access to issuance and verification history. Each credential is a deployment of the
\verb|Credential| contract, and each grant of the credential is a \verb|CredentialGrant| that the
grantee can access and manage verification requests with. In addition, \verb|VerifyRequest| also stores each request as a smart contract deployment to make tracking attempts of identity theft easier.

\subsection{ResilientDB Modifications}

ResilientDB’s contract service depends on an archived implementation of EVM by Microsoft called
eEVM. While EVM itself isn’t specifically aware of complex data types like strings, the Solidity
compiler is aware of the ABI and properly implements string handling as bytecode instructions.
However, ResilientDB contract service does not implement the full ABI and treats function parameters
as a series of \verb|uint256|, which is severely limiting.

To work around the restriction, we have implemented support for accepting strings as function
arguments and appending them to EVM input byte vectors. We have implemented a custom argument parser
and serializer for the contract manager that supports both \verb|uint256| and \verb|string| arguments. This custom serializer is applied to both the contract deployment and execution stages, meaning that all functions, including the constructor, can use strings as inputs.

\subsection{Roadblocks} \label{ssec:Roadblocks}

We have encountered tremendous difficulty in deploying the full \verb|Registry| contract onto ResilientDB due to problematic implementation of the underlying eEVM library, which is not continued any more and also only targets the oldest EVM spec (Homestead), a version that does not support many of the new features that make Solidity development more ergonomic. In addition, eEVM does not support all instructions in Homestead, such as REVERT (opcode 0xfd).

What is most damaging, however, is the inability to deploy/instantiate another contract with functions inside an existing contract, which is a basic feature that is allegedly available in Homestead (based on our research and lack of \verb|solc| compile errors when targeting Homestead). In an attempt to ensure full functionality of \verb|Registry| smart contract APIs, we have tried multiple workarounds that produced fewer errors but does not fully resolve all issues such as not using public variables (i.e., variables with an implicit eponymous view function), using constructor with no arguments, using alternative functions other than the constructor to initialize the state of the contract, etc. We have also tried disassembling the bytecode to try to identify the problematic opcode that causes REVERTs even though no verb|require| is specified, but this task proved to be outside the scope of the class. As we were unable to overcome it, this limitation resulted in a partial rewrite of the smart contract.

Some other roadblocks are caused by the rudimentary use of eEVM on the smart contract service’s
part. For instance, \verb|AddressManager::CreateContractAddress(const Address&)| uses a fixed zero
nonce when creating contract addresses, which causes an address to only be able to deploy one
contract; we fixed this by using \verb|rand()| as a nonce. Also as mentioned earlier, the smart
contract service does not support the EVM ABI fully, which means we cannot access string parameters
natively. 

One other particular problem we faced was the lack of \verb|array| support. The GraphQL backend does
not allow for any types to be passed in as parameters or returned except for basic primitives. This
forced us to find a creative method of passing back a list of contracts from the backend through the
usage of ID numbers, which are of type \verb|uint256|.

We believe that the eEVM implementation used by ResilientDB is inherently flawed and incomplete, and it should be replaced with a more updated version of EVM as soon as possible in consideration of the longevity of ResilientDB—soon \verb|solc| will not be able to target EVM specs older than Constantinople.

