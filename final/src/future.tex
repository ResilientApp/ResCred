\section{Future Work and advice}

We stronly recommend future students taking ECS189F who wish to continue this project consider the possibility of substituting the underlying EVM implementation, or pivoting to an architecture where more involved smart contract capabilities are not needed for the tech stack to function as a whole.
In fact, any future projects considering planning to utilize the SmartContract CLI should consider
making their project on modifying, updating, and documenting SmartContract.
From our experience interacting and working with the current SmartContract interface, we have found
some significant flaws as mentioned in \autoref{ssec:Roadblocks}.

Regarding this specific project, our team believes that we have maximimized our efforts in
delivering a semi-working deliverable on our original objective of creating a decentralized
credential storage and verification system. However, there are a number of issues that we have
discovered along the way that we would \textit{like} to fix but simply have no time to implement
them. One such item is security. We believe the original smart contract service is not particularly
resilient to invalid inputs and may have security vulnerabilities within the codebase that might
allow attackers to forge contracts, defeating the purpose of such a system. That is, of course, not
mentioning the risk of crashing the server on invalid inputs, forcing us to be very meticulus on
what inputs we are passing into the CLI. Furthermore, given our
time constraints this quarter, we actually disabled certain security features to allow our code to
run more compatibly with the Smart Contract CLI. As such, one direction we would like to bring this
project to is to secure and harden the Smart Contract CLI and our ResCred system. 
