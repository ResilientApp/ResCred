\section{Problem}

Due to the availability and access to technologies like Photoshop, tampering and editing of
credentials has become a worldwide issue. In fact, a study recently found that 75\% of admission staff were unable to identify a fake credential. With increasing digital credentials and a global trend in the rise of remote work, verifying the legitimacy of qualifications has become even more critical. On a day to day basis, companies use credentials like employee badges, university diplomas, and government IDs to verify claims about a person's identity and qualifications.

The traditional system of verifying credentials relies on centralized institutions (universities, government bodies, companies etc.) and manual processes, which are often time-consuming, prone to error, and vulnerable to manipulations. A scalable and tamper-proof database would be able to provide transparent access and verification of credentials by the general public, including employers.

